\documentstyle[11pt]{article}
\setlength{\topmargin}{-.5in}
\setlength{\textheight}{9in}
\setlength{\oddsidemargin}{.125in}
\setlength{\textwidth}{6.25in}
\begin{document}
\title{litcomp}
\author{Andy Christianson}
\renewcommand{\today}{March 13, 2011}
\maketitle

This article exhibits the design of and motivation for a literate
programming platform named {\em litcomp}.

\section {Synopsis}

The litcomp platform centers around the concept of defining programs primarly
in the form of human-friendly literature. Rather than defining programs
as a hierarchy of computer-friendly source files, programs are defined
as a web of ideas, connected in a deliberate, thought-out manner.

The canonical literate programming tools, written by D.E. Knuth, are
{\em web} and {\em tangle}. These programs are language-agnostic and are
sufficient tools in defining and compiling literate programs. Although
these programs satisfy the essential needs of a literate programmer,
they were not designed to accomodate users preferring web-based
interaction. The programs do not leverage the power of the class of tools
whose primary purpose is interacting with webs of content: web browsers.

Granting that web browsers are the best tool for user interaction with
web content, that literate programs are structured in precisely this
format, and that there does not yet exist a web-based literate programming
platform, the need for such a platform is apparent.

\end{document}
