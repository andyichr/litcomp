\documentstyle[11pt]{article}
\setlength{\topmargin}{-.5in}
\setlength{\textheight}{9in}
\setlength{\oddsidemargin}{.125in}
\setlength{\textwidth}{6.25in}
\begin{document}
\title{litcomp}
\author{Andy Christianson}
\renewcommand{\today}{March 13, 2011}
\maketitle

This article exhibits the design of and motivation for a literate
programming platform named {\em litcomp}.

\section {Synopsis}

The litcomp platform centers around the concept of defining programs primarly
in the form of human-friendly literature. Rather than defining programs
as a hierarchy of computer-friendly source files, programs are defined
as a web of ideas, connected in a deliberate, thought-out manner.

The canonical literate programming tools, written by D.E. Knuth, are
{\em web} and {\em tangle}. These programs are language-agnostic and are
sufficient tools in defining and compiling literate programs. Although
these programs satisfy the essential needs of a literate programmer,
they were not designed to accomodate users preferring web-based
interaction. The programs do not leverage the power of the class of tools
whose primary purpose is interacting with webs of content: web browsers.

Granting that web browsers are the best tool for user interaction with
web content, that literate programs are structured in precisely this
format, and that there does not yet exist a web-based literate programming
platform, the need for such a platform is apparent.

\section {Architecture}

litcomp has two essential top-level components: model and interface.

\subsection {Model}

The litcomp model is a category containing objects relevant to
the motivation of the literate program under operation. In some cases
the objects of the category are defined explicitly while in other cases
only morphism and invariants are defined. Generally it is desirable to
define the program under operation as concisely and clearly as possible.

Whether a given object is explicitly defined or derived by demand is a
function of its classification. In this context, objects of the category
are distributed between two mutually-exclusive classes:

\begin{itemize}

\item {\em Essential Objects}: the smallest set of objects and morphisms
from which it is possible to derive all relevant objects

\item {\em Derivative Objects}: objects not classified as essential

\end{itemize}

Essential objects are always defined explicitly in a persistent data
store whereas derivative objects are generally only defined temporarily
in volatile storage.

\subsection {Interface}

The primary user interface to the litcomp platform is a web
application. Specifically, this application consists of a HTTP-compliant
web server defined in JavaScript and a set of HTML documents and
client-side JavaScript programs. Server-side JavaScript programs are
evaluated by node.js.

\end{document}
